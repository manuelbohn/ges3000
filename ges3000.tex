\documentclass[man]{apa6}

\usepackage{amssymb,amsmath}
\usepackage{ifxetex,ifluatex}
\usepackage{fixltx2e} % provides \textsubscript
\ifnum 0\ifxetex 1\fi\ifluatex 1\fi=0 % if pdftex
  \usepackage[T1]{fontenc}
  \usepackage[utf8]{inputenc}
\else % if luatex or xelatex
  \ifxetex
    \usepackage{mathspec}
    \usepackage{xltxtra,xunicode}
  \else
    \usepackage{fontspec}
  \fi
  \defaultfontfeatures{Mapping=tex-text,Scale=MatchLowercase}
  \newcommand{\euro}{€}
\fi
% use upquote if available, for straight quotes in verbatim environments
\IfFileExists{upquote.sty}{\usepackage{upquote}}{}
% use microtype if available
\IfFileExists{microtype.sty}{\usepackage{microtype}}{}

% Table formatting
\usepackage{longtable, booktabs}
\usepackage{lscape}
% \usepackage[counterclockwise]{rotating}   % Landscape page setup for large tables
\usepackage{multirow}		% Table styling
\usepackage{tabularx}		% Control Column width
\usepackage[flushleft]{threeparttable}	% Allows for three part tables with a specified notes section
\usepackage{threeparttablex}            % Lets threeparttable work with longtable

% Create new environments so endfloat can handle them
% \newenvironment{ltable}
%   {\begin{landscape}\begin{center}\begin{threeparttable}}
%   {\end{threeparttable}\end{center}\end{landscape}}

\newenvironment{lltable}
  {\begin{landscape}\begin{center}\begin{ThreePartTable}}
  {\end{ThreePartTable}\end{center}\end{landscape}}

  \usepackage{ifthen} % Only add declarations when endfloat package is loaded
  \ifthenelse{\equal{\string man}{\string man}}{%
   \DeclareDelayedFloatFlavor{ThreePartTable}{table} % Make endfloat play with longtable
   % \DeclareDelayedFloatFlavor{ltable}{table} % Make endfloat play with lscape
   \DeclareDelayedFloatFlavor{lltable}{table} % Make endfloat play with lscape & longtable
  }{}%



% The following enables adjusting longtable caption width to table width
% Solution found at http://golatex.de/longtable-mit-caption-so-breit-wie-die-tabelle-t15767.html
\makeatletter
\newcommand\LastLTentrywidth{1em}
\newlength\longtablewidth
\setlength{\longtablewidth}{1in}
\newcommand\getlongtablewidth{%
 \begingroup
  \ifcsname LT@\roman{LT@tables}\endcsname
  \global\longtablewidth=0pt
  \renewcommand\LT@entry[2]{\global\advance\longtablewidth by ##2\relax\gdef\LastLTentrywidth{##2}}%
  \@nameuse{LT@\roman{LT@tables}}%
  \fi
\endgroup}


\ifxetex
  \usepackage[setpagesize=false, % page size defined by xetex
              unicode=false, % unicode breaks when used with xetex
              xetex]{hyperref}
\else
  \usepackage[unicode=true]{hyperref}
\fi
\hypersetup{breaklinks=true,
            pdfauthor={},
            pdftitle={The title},
            colorlinks=true,
            citecolor=blue,
            urlcolor=blue,
            linkcolor=black,
            pdfborder={0 0 0}}
\urlstyle{same}  % don't use monospace font for urls

\setlength{\parindent}{0pt}
%\setlength{\parskip}{0pt plus 0pt minus 0pt}

\setlength{\emergencystretch}{3em}  % prevent overfull lines


% Manuscript styling
\captionsetup{font=singlespacing,justification=justified}
\usepackage{csquotes}
\usepackage{upgreek}

 % Line numbering
  \usepackage{lineno}
  \linenumbers


\usepackage{tikz} % Variable definition to generate author note

% fix for \tightlist problem in pandoc 1.14
\providecommand{\tightlist}{%
  \setlength{\itemsep}{0pt}\setlength{\parskip}{0pt}}

% Essential manuscript parts
  \title{The title}

  \shorttitle{Title}


  \author{First Author\textsuperscript{1}~\& Ernst-August Doelle\textsuperscript{1,2}}

  % \def\affdep{{"", ""}}%
  % \def\affcity{{"", ""}}%

  \affiliation{
    \vspace{0.5cm}
          \textsuperscript{1} Wilhelm-Wundt-University\\
          \textsuperscript{2} Konstanz Business School  }

  \authornote{
    Add complete departmental affiliations for each author here. Each new
    line herein must be indented, like this line.
    
    Enter author note here.
    
    Correspondence concerning this article should be addressed to First
    Author, Postal address. E-mail:
    \href{mailto:my@email.com}{\nolinkurl{my@email.com}}
  }


  \abstract{Enter abstract here. Each new line herein must be indented, like this
line.}
  \keywords{keywords \\

    \indent Word count: X
  }





\usepackage{amsthm}
\newtheorem{theorem}{Theorem}[section]
\newtheorem{lemma}{Lemma}[section]
\theoremstyle{definition}
\newtheorem{definition}{Definition}[section]
\newtheorem{corollary}{Corollary}[section]
\newtheorem{proposition}{Proposition}[section]
\theoremstyle{definition}
\newtheorem{example}{Example}[section]
\theoremstyle{definition}
\newtheorem{exercise}{Exercise}[section]
\theoremstyle{remark}
\newtheorem*{remark}{Remark}
\newtheorem*{solution}{Solution}
\begin{document}

\maketitle

\setcounter{secnumdepth}{0}



Bruner (1974)

\section{General setup and procedure}\label{general-setup-and-procedure}

\section{Part 1.1: Spontaneous production of
gestures}\label{part-1.1-spontaneous-production-of-gestures}

\subsection{Participants}\label{participants}

\subsection{Design and Procedure}\label{design-and-procedure}

\subsection{Results and discussion}\label{results-and-discussion}

\section{Part 1.2: Uptake}\label{part-1.2-uptake}

\subsection{Participants}\label{participants-1}

\subsection{Design and Procedure}\label{design-and-procedure-1}

\subsection{Results and discussion}\label{results-and-discussion-1}

\section{Part 1.3: Comprehension}\label{part-1.3-comprehension}

\subsection{Participants}\label{participants-2}

\subsection{Design and Procedure}\label{design-and-procedure-2}

\subsection{Results and discussion}\label{results-and-discussion-2}

\section{Part 1.4: Abstract concepts (empty
picture)}\label{part-1.4-abstract-concepts-empty-picture}

\subsection{Participants}\label{participants-3}

\subsection{Design and Procedure}\label{design-and-procedure-3}

\subsection{Results and discussion}\label{results-and-discussion-3}

\section{Part 1.5: Convergence}\label{part-1.5-convergence}

\subsection{Participants}\label{participants-4}

\subsection{Design and Procedure}\label{design-and-procedure-4}

\subsection{Results and discussion}\label{results-and-discussion-4}

\section{Part 1.6: Drift to
arbitrary}\label{part-1.6-drift-to-arbitrary}

\subsection{Production}\label{production}

\subsubsection{Participants}\label{participants-5}

\subsubsection{Design and Procedure}\label{design-and-procedure-5}

\subsubsection{Results and discussion}\label{results-and-discussion-5}

\subsection{Comprehension}\label{comprehension}

\subsubsection{Participants}\label{participants-6}

\subsubsection{Design and Procedure}\label{design-and-procedure-6}

\subsubsection{Results and discussion}\label{results-and-discussion-6}

\section{Part 2.1: Size}\label{part-2.1-size}

\subsection{Participants}\label{participants-7}

\subsection{Design and Procedure}\label{design-and-procedure-7}

\subsection{Results and discussion}\label{results-and-discussion-7}

\section{Part 2.2: Number}\label{part-2.2-number}

\subsection{Participants}\label{participants-8}

\subsection{Design and Procedure}\label{design-and-procedure-8}

\subsection{Results and discussion}\label{results-and-discussion-8}

\section{Part 2.3: Movement}\label{part-2.3-movement}

\subsection{Participants}\label{participants-9}

\subsection{Design and Procedure}\label{design-and-procedure-9}

\subsection{Results and discussion}\label{results-and-discussion-9}

\section{Part 2.4: Transitive actions with two
agents}\label{part-2.4-transitive-actions-with-two-agents}

\subsection{Participants}\label{participants-10}

\subsection{Design and Procedure}\label{design-and-procedure-10}

\subsection{Results and discussion}\label{results-and-discussion-10}

\section{Part 2.5: Transitive actions with three
agents}\label{part-2.5-transitive-actions-with-three-agents}

\subsection{Participants}\label{participants-11}

\subsection{Design and Procedure}\label{design-and-procedure-11}

\subsection{Results and discussion}\label{results-and-discussion-11}

\section{Part 2 overall analysis}\label{part-2-overall-analysis}

\subsection{Participants}\label{participants-12}

\subsection{Design and Procedure}\label{design-and-procedure-12}

\subsection{Results and discussion}\label{results-and-discussion-12}

\newpage

\section{References}\label{references}

\setlength{\parindent}{-0.5in} \setlength{\leftskip}{0.5in}

\hypertarget{refs}{}
\hypertarget{ref-bruner1974communication}{}
Bruner, J. S. (1974). From communication to language: A psychological
perspective. \emph{Cognition}, \emph{3}(3), 255--287.
doi:\href{https://doi.org/10.1016/0010-0277(74)90012-2}{10.1016/0010-0277(74)90012-2}






\end{document}
